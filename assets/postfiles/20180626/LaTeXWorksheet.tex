\documentclass{article}
\usepackage{hyperref,lipsum,amsmath,amssymb}
\hypersetup{colorlinks=true} % make sure our hyperlinks are coloured for visibility
% We define an Author, Title and Date
\author{\emph{Your Name Here}}
\title{A Short \LaTeX{} Worksheet \\ \small Created by Sam Fearn -- s.m.fearn@durham.ac.uk}
\date{March 15\textsuperscript{th}, 2018}

\begin{document}
% Create a title from our Author, Title and Date
\maketitle
% We put our content into section environments
\section{Introduction}
\label{sec:introduction}
% I didn't have time to discuss labels today, but adding labels to your document allows you to refer back to the relevant place later in your document using the \ref{label} command. For example, I could refer to the introduction later in my document by saying `see section \ref{sec:introduction}'. This would produce the output `see section 1'. The advantage os using labels, is that if I add a reference to section 2, and then later add a new section before the old section 2, the reference will automatically be updated to say section 3. Try adding some references to the various sections and subsections of this document.
This worksheet has been created using the \LaTeX{} typesetting system. By reproducing this document you should be able to develop your skills with \LaTeX{}. You should \emph{typeset} your file regularly to see whether it is looking correct. Try it now if you haven't already. If everything has worked properly you should now have a PDF file of the worksheet up to this point. 

\section{Content}
\label{sec:content}

\emph{Sections} allow us to add structure our documents. For a short document such as this, we will use \emph{sections} and \emph{subsections}. A longer document such as a thesis would also use \emph{chapters} (and the \emph{report} document class). The first this we will discuss is how to typeset mathematics. To start with, ignore the `verbatim' code and hyperlinks in the following subsection, these will be explained later. A good reference for some of the symbols and commands in the following subsections is \href{https://wch.github.io/latexsheet/latexsheet.pdf}{the \LaTeXe{} cheat sheet}.

\subsection{Typesetting Mathematics}
\label{sub:typesetting_mathematics}

One of the reasons to use \LaTeX{} is that it can produce very good looking mathematical formulae. We have already seen some examples in the talk. \LaTeX{} only recognises mathematical commands while in `maths mode'. There are a few ways to enter maths mode -- which one we use depends on how we want the output to be formatted. If I want the maths to be `inline', part of the current line, then I place it between dollar symbols. Therefore \$ x = 4 \$ produces $x=4$. If I want to produce an equation which is placed on its own line then I can use the \emph{\textbf{equation} environment}.
\begin{verbatim}
	\begin{equation}
	    x^3 + 2 \ge 6.
	\end{equation}
\end{verbatim}
This produces
\begin{equation}
	x^3 + 2 \ge 6.
\end{equation}
See \href{https://www.sharelatex.com/learn/Mathematical_expressions}{this webpage} for more information on mathematics modes. Try to reproduce the rest of this subsection as closely as you can.

A \emph{bijection} is a map between two sets which is both \emph{injective} and \emph{surjective}. Let us now define these terms. Given two sets $A$ and $B$, a map $\phi:A \to B$ is said to be injective if
\begin{equation}\label{eq:injective} % I can also add referenecs to my equations using labels.
	\phi(a_1) = \phi(a_2) \iff a_1 = a_2,\ \forall\ a_1,a_2 \in A.
\end{equation}
Furthermore, the map is surjective if
\begin{equation}\label{eq:surjective}
	\forall\ b \in B,\ \exists\ a \in A\ |\ \phi(a)=b.
\end{equation}
% Try uncommenting the following line.
% If both equation \ref{eq:injective} and equation \ref{eq:surjective} are satisfied, then the map is said to be \emph{bijective}.

The talk also covered the import result of the \emph{Gaussian integral}
\begin{equation}
	\int_{-\infty}^\infty e^{-\frac{1}{2}\left( \frac{x - \mu}{\sigma} \right)^2} dx = \sigma\sqrt{2 \pi}.
\end{equation}

In the partitions project, we saw how the geometric progression formula
\begin{equation}
	\frac{1}{1-x} = 1 + x + x^2 + \ldots + x^n + \ldots = \sum_{n=0}^\infty x^n,
\end{equation}
could be used to deduce the generating function for the partition number $p(n)$,
\begin{equation}
	\sum_{n=0}^\infty p(n) x^n = \prod_{n=1}^\infty \frac{1}{1-x^n}.
\end{equation}

\subsection{Verbatim} % (fold)
\label{sub:verbatim}

In order to explain some of the commands that you will need to add to your file, we will need to show the names of the commands on the worksheet. In order for \LaTeX{} to display the command rather than process it we use an \emph{environment} called \textbf{Verbatim}.
\begin{verbatim}
	\begin{verbatim}
	    \LaTeX{} is now printed rather than processed.
	\end{verbatim }
\end{verbatim}
% Note the space after in the inner \end{verbatim }. This is so \LaTeX{} understand the difference between the one I want to print verbatim and the actual end of the verbatim environment. 
We can also produce verbatim output in the middle of a line using \verb!\verb|\LaTeX{}|!.
% We can use both the exclamation mark or the pipe mark to indicate what content is to be inline verbatim. As previously, I have to use one of each to ensure that \LaTeX{} understands which one ends the actual verb environment, and which is to be printed verbatim.
\subsection{Packages}
\label{sub:packages}

One of the advantages of \LaTeX{} (or \TeX{}) is that it can be extended using \emph{packages}. A package is \emph{loaded} by adding the the line
\begin{verbatim}
	\usepackage{package_name}
\end{verbatim}
into the \emph{preamble} of your tex file. The preamble is the part of the file before the line \verb|\begin{document}|. Everything between \verb|\begin{document}| and \verb|\end{document}| is called the \emph{body} of the document. We can use a package called \href{https://ctan.org/pkg/lipsum?lang=en}{lipsum} to produce dummy text. This hyperlink was in turn created with a package called \href{latex hyperref}{hyperref} and the command used was
\begin{verbatim}
	\href{https://ctan.org/pkg/lipsum?lang=en}{lipsum}.
\end{verbatim}
The following paragraph is produce using the lipsum package, you can read about how to use this package from the package documentation on the previously linked webpage for lipsum -- the following is the lipsum paragraph 66.

\lipsum[66]

Two packages that we will often want to use are the \href{https://michael-prokop.at/latextug/amsldoc.pdf}{amsmath and amssymb} packages, which are documented at the previously linked page. They add more mathematical commands that you can then use in you documents, such as matrices, unnumbered environments and mathematical fonts. Once I have loaded amsmath and amssymb, I can prodcue the output
\begin{equation*}
	A = \begin{pmatrix}
		4 - 2i & i \\
		-i & a
	\end{pmatrix} \in M_2(\mathbb{C}).
\end{equation*}
These packages also allow us to align equations using the \textbf{align} environment. We can use this to define two block matrices $B$ and $C$. We can then give the multiplication of these matrices in a new equation environment. If
\begin{align}
	B &= \left(\begin{array}{c | c} % The c here is to centre align the array columns.
		P & Q \\
		\hline
		R & S
	\end{array}\right),\nonumber \\
	C &= \left(\begin{array}{c | c}
		W & X \\
		\hline
		Y & Z
	\end{array}\right),
\end{align}
then
\begin{equation}
	BC = \left(\begin{array}{c | c}
		PW + QY & PX + QZ \\
		\hline
		RW + SY & RX + SZ
	\end{array}\right).
\end{equation}
% There's quite a lot going on in the previous few lines, this was supposed to be the hardest part to reproduce for anyone who already had some experience with latex. The \left( and \right) commands are used to produce parentheses which scale automatically based on what is inside them. The array environment lets me format content in columns and specify a serperator. I use this to produce the vertical line of the block matrix, the horizontal line is produced with the \hline command. The ampersand character & is used by \LaTeX{} to align elements. It is used between the B and the equals and between the C and the equals to align these two lines. The double slash \\ after the first right) is used to end the first line and stat a new line. The ampersand is also used inside the array environment to tell \LaTeX{} how to align the elements of the columns. Finally, the \nonumber command suppresses the equation number for the first line of the align environment.

\section{Conclusion}
\label{sec:conclusion}
Regardless of how much of this sheet you have been able to reproduce, I hope you have learnt something about \LaTeX{}. In my opinion, it is best to learn \LaTeX{} simply by starting to use it, and answering any questions you have as you go by using the many excellent resources online. The \href{https://www.sharelatex.com/learn}{ShareLaTeX} website contains a lot of useful help, as does \href{https://tobi.oetiker.ch/lshort/lshort.pdf}{The Not So Short Introduction To \LaTeXe{}}. The tex file used to produce this talk will be available shortly so that you can compare with your tex file. I have also added some comments to my file to explain some additional things, so I encourage you to read through if you are interested.

\end{document}
